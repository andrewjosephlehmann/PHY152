% !TeX program = lualatex

\documentclass{article}
\usepackage{tikz}
\usepackage[utf8]{inputenc}
\setlength{\parindent}{0em} %indents to paragraphs
\setlength{\parskip}{1em} %lineskips after paragraph breaks
\usepackage[margin=1.0in]{geometry} %margins
\usepackage{bbm}
\usepackage{amsmath}
\usepackage{amsthm}
\usepackage{enumerate}
\usepackage{mathtools}
\usepackage{amssymb}
\usepackage{tikz}
\usepackage{tikz-feynman}
\usepackage{amsmath}
\usepackage{physics}
\usetikzlibrary{calc}
\usepackage{feynmp}
\usepackage{feynmp-auto}
\usepackage{breqn}
\usepackage{graphicx}
\usepackage{mathtools}
\usetikzlibrary{decorations.pathmorphing,patterns}
\newcommand{\vectorproj}[2][]{\textit{proj}_{{#1}}{#2}}
\newcommand{\vect}{\mathbf}
\usepackage{pgfplots}
\usetikzlibrary{intersections,angles,quotes}
\usepgfplotslibrary{fillbetween}
\usetikzlibrary{decorations.pathmorphing,patterns}
\tikzstyle{spring}=[thick,decorate,decoration={zigzag,pre length=0.1cm,post
	length=0.1cm,segment length=6}]

\newenvironment{amatrix}[1]{%
	\left(\begin{array}{@{}*{#1}{c}|c@{}}
	}{%
	\end{array}\right)
}
\makeatletter
\let\@@span\span
\def\sp@n{\@@span\omit\advance\@multicnt\m@ne}
\makeatother

\renewcommand{\span}{...}
\newcommand{\galaxy}{\includegraphics[width=0.13in]{image}}


\title{PHY152: Problem Set 6}
\author{A. Lehmann}
\date{March 12th 2023}

\begin{document}
	\maketitle
	\tableofcontents
\newpage
\begin{center}
	$ x(t)=Acos\left(\frac{2\pi t}{T}\right)\leftrightarrow x(t)=Acos\left(2\pi ft\right)
	$
\end{center}

\section{PART A}
\subsection{Cosine Wave}
\textbar{The graph provided by \textit{Mastering Physics} for PS6A illustrates the position of an object as a function of time. Let us derive the equation for the oscillation of this graph. 
	
	\begin{center}
		\centering
		\includegraphics[width=0.5\linewidth]{../../../../../../../../../Downloads/MHM_xt}
		\label{fig:mhmxt}
	\end{center}
\textbar{Before determining the values of $M,N$, and $T$ let us derive a solution to simple harmonic motion. }

\subsubsection{Derivation of SHM}
Simple harmonic motion is a periodic motion in which the object moves back and forth along a straight line. It's deviation from equilibrium  is proportional to the force acting on it. }
\begin{center}
	\textit{Spring force formula:}
	$ F = k(x-x_{0}) = -kx$
\end{center}
\textbar{Now apply Newtons second law:}
\begin{center}
	$ m\vec{a} = -kx
	$
\end{center}
\textbar{Acceleration is the derivative of velocity with respect to time, and velocity is the derivative of position with respect to time. Thus, we may rewrite acceleration as:}

\begin{center}
	$a= \frac{dv}{dt} = \frac{d^{2}x}{dt^2}		
	$
\end{center}
\newpage
\textbar{Since equating Newton's second law with spring force is a differential equation, we shall rearrange it in terms of its derivatives:}
\begin{center}
	$ m\left(\frac{d^{2}x}{dt^2}\right) = -kx \leftrightarrow m\left(\frac{d^{2}x}{dt^2}\right) + kx = 0
	$
\end{center}

\textbar{To describe simple harmonic motion, I will establish equivalence between to the x-coordinates of a particle moving in uniform circular motion.}

\textit{$P$ = Position of Arbitrary Point-Particle \\
	$A$ = Amplitude \\
	$\theta$ = Angle with respect from x-axis to P}
\begin{center}
\begin{tikzpicture}
	\tikzset{>=stealth}
	% draw axis and labels. We store a single coordinate to have the
	% direction of the x axis
	\draw[->] (-4,0) -- ++(8,0) coordinate (X) node[below] {$x$};
	\draw[->] (0,-4) -- ++(0,8) node[left] {$y$};
	
	\newcommand\CircleRadius{3cm}
	\draw (0,0) circle (\CircleRadius);
	% special method of noting the position of a point
	\coordinate (P) at (45:\CircleRadius);
	
	\draw[very thick] 
	(0,0) 
	coordinate (O) % store origin
	node[below right] {} % label
	-- 
	node[above right,pos=1] {$P$} % some labels
	node[above left,midway] {$A$}
	(P) 
	-- 
	node[midway,right] {$A\sin\theta$}
	(P |- O) coordinate (Px) % projection onto horizontal line through
	% O, saved for later
	-- 
	node[midway,below] {$A\cos\theta$}
	cycle % closed path
	% pic trick is from the angles library, requires the three points of
	% the marked angle to be named
	pic [draw,blue,angle radius=1cm,pic text=$\theta$,
	angle eccentricity=1.3] {angle=X--O--P};
	
	% right angle marker
	\draw ($(Px)-(0.3,0)$) -- ++(0,0.3) -- ++(0.3,0);
	%Point Particle drawing
	\filldraw[blue] (2.12,2.12) circle (4pt);
	
\end{tikzpicture}
\end{center}

\textbar{Therefore, we may write the position over time as:}
\begin{center}
$ x(t)=A\cos(\omega t) \implies x(t) = A\cos(\omega t + \theta_{0})
$
\end{center}	
\textbar{Differentiating this equation twice, we find:}

\begin{center}
$ \frac{dx}{dt} = -A\omega \sin(\omega t)
$
\end{center}
\begin{center}
	$\frac{d^{2}x}{dt^2} = - A\omega^2\cos(\omega t)
	$
\end{center}

\textbar{Inserting the equation for acceleration and position about the x-axis into our differential equation yields:}

\begin{center}
$m(- A\omega^2\cos(\omega t)) + k(A\cos(\omega t)) = 0 
$
\end{center}

\begin{center}
	$- A\omega^2\cos(\omega t) + \frac{k}{m}(A\cos(\omega t)) = 0 
	$
\end{center}
\newpage
\textbar{For when $\omega = \sqrt{\frac{k}{m}}$ then:}
\begin{center}
$ x = A\cos\left(t\sqrt{\frac{k}{m}}\right) \implies x = A\cos\theta
$
\end{center}























%	\begin{tikzpicture}[black!75,thick]
		
	%			\filldraw[blue] (13,0) circle (4pt);
	%	% Supporting structure
%		\fill [pattern = north west lines] (3,-1.5) rectangle ++(-.2,3);
%		\draw[thick] (3,-1.5) -- ++(0,3);
		
%		\draw
%		[
%		decoration={
%			coil,
%			segment length = 1mm,
%			amplitude = 2mm,
%			aspect = 0.5,
%			post length = 3mm,
%			pre length = 3mm},
%		decorate] (3,0) -- ++(10,0);
		

%	\end{tikzpicture}





\begin{center}
	\centering
	\includegraphics[width=0.5\linewidth]{../../../../../../../../../Downloads/MHM_xt}
	\label{fig:mhmxt}
\end{center}
\subsubsection{Unknown constants $M,$ $N$ and $T$}
\textit{The graph is described by following equation:}
\begin{center}
	$x(t) = Acos(\omega t + \theta_0)
	$
\end{center}
\textbar{By observing the graph, we may determine that the contant $M$ is equivalent to the amplitude of the function. The amplitude is the value which dictates the maximum value of $x(t)$.
	
To uncover angular velocity $\omega$:}

\begin{center}
	$A\cos(\omega t + \theta_{0}) = A\cos(\omega(t+T)+\theta_0)
	$

$  \omega t + \theta_0 + 2\pi = \omega(t+T) + \theta_0
$

$  2\pi = \omega T
$
\end{center}
\textit{Solve, for $\omega$:}
\begin{center}
	$ \omega = \frac{2\pi}{T}
	$
\end{center}
\textbar{To uncover $\theta_0$:}
\begin{center}
	$ \omega = \frac{2\pi}{T} 
$

	$ \frac{\theta_0}{t} = \frac{2\pi}{T}
	$

	$\theta_0 = \frac{2\pi(t)}{T}
	$
	
	$\theta_0 = \frac{2\pi(N)}{T}
	$
\end{center}
\section{PART B} 


\end{document}
