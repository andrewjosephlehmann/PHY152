% !TeX program = lualatex

\documentclass{article}
\usepackage{tikz}
\usepackage[utf8]{inputenc}
\setlength{\parindent}{0em} %indents to paragraphs
\setlength{\parskip}{1em} %lineskips after paragraph breaks
\usepackage[margin=1.0in]{geometry} %margins
\usepackage{bbm}
\usepackage{amsmath}
\usepackage{amsthm}
\usepackage{enumerate}
\usepackage{mathtools}
\usepackage{amssymb}
\usepackage{tikz}
\usepackage{tikz-feynman}
\usepackage{amsmath}
\usepackage{physics}
\usetikzlibrary{calc}
\usepackage{feynmp}
\usepackage{feynmp-auto}
\usepackage{breqn}
\usepackage{graphicx}
\usepackage{mathtools}
\usetikzlibrary{decorations.pathmorphing,patterns}
\newcommand{\vectorproj}[2][]{\textit{proj}_{{#1}}{#2}}
\newcommand{\vect}{\mathbf}
\usepackage{pgfplots}
\usetikzlibrary{intersections,angles,quotes}
\usepgfplotslibrary{fillbetween}
\usetikzlibrary{decorations.pathmorphing,patterns}
\tikzstyle{spring}=[thick,decorate,decoration={zigzag,pre length=0.1cm,post
	length=0.1cm,segment length=6}]

\newenvironment{amatrix}[1]{%
	\left(\begin{array}{@{}*{#1}{c}|c@{}}
	}{%
	\end{array}\right)
}
\makeatletter
\let\@@span\span
\def\sp@n{\@@span\omit\advance\@multicnt\m@ne}
\makeatother

\renewcommand{\span}{...}
\newcommand{\galaxy}{\includegraphics[width=0.13in]{image}}


\title{PHY152: Problem Set 6}
\author{A. Lehmann}
\date{March 12th 2023}

\begin{document}
	\maketitle
	\tableofcontents
\newpage
\begin{center}
	$ x(t)=Acos\left(\frac{2\pi t}{T}\right)\leftrightarrow x(t)=Acos\left(2\pi ft\right)
	$
\end{center}

\section{PART A}
\subsection{Cosine Wave}
\textbar{The graph provided by \textit{Mastering Physics} for PS6A illustrates the position of an object as a function of time. Let us derive the equation for the oscillation of this graph. 
	
	\begin{center}
		\centering
		\includegraphics[width=0.5\linewidth]{../../../../../../../../../Downloads/MHM_xt}
		\label{fig:mhmxt}
	\end{center}
\textbar{Before determining the values of $M,N$, and $T$ let us derive a solution to simple harmonic motion. }

\subsubsection{Derivation of SHM}
Simple harmonic motion is a periodic motion in which the object moves back and forth along a straight line. It's deviation from equilibrium  is proportional to the force acting on it. }
\begin{center}
	\textit{Spring force formula:}
	$ F = k(x-x_{0}) = -kx$
\end{center}
\textbar{Now apply Newtons second law:}
\begin{center}
	$ m\vec{a} = -kx
	$
\end{center}
\textbar{Acceleration is the derivative of velocity with respect to time, and velocity is the derivative of position with respect to time. Thus, we may rewrite acceleration as:}

\begin{center}
	$a= \frac{dv}{dt} = \frac{d^{2}x}{dt^2}		
	$
\end{center}
\newpage
\textbar{Since equating Newton's second law with spring force is a differential equation, we shall rearrange it in terms of its derivatives:}
\begin{center}
	$ m\left(\frac{d^{2}x}{dt^2}\right) = -kx \leftrightarrow m\left(\frac{d^{2}x}{dt^2}\right) + kx = 0
	$
\end{center}

\textbar{To describe simple harmonic motion, I will establish equivalence between to the x-coordinates of a particle moving in uniform circular motion.}

\textit{$P$ = Position of Arbitrary Point-Particle \\
	$A$ = Amplitude \\
	$\theta$ = Angle with respect from x-axis to P}
\begin{center}
\begin{tikzpicture}
	\tikzset{>=stealth}
	% draw axis and labels. We store a single coordinate to have the
	% direction of the x axis
	\draw[->] (-4,0) -- ++(8,0) coordinate (X) node[below] {$x$};
	\draw[->] (0,-4) -- ++(0,8) node[left] {$y$};
	
	\newcommand\CircleRadius{3cm}
	\draw (0,0) circle (\CircleRadius);
	% special method of noting the position of a point
	\coordinate (P) at (45:\CircleRadius);
	
	\draw[very thick] 
	(0,0) 
	coordinate (O) % store origin
	node[below right] {} % label
	-- 
	node[above right,pos=1] {$P$} % some labels
	node[above left,midway] {$A$}
	(P) 
	-- 
	node[midway,right] {$A\sin\theta$}
	(P |- O) coordinate (Px) % projection onto horizontal line through
	% O, saved for later
	-- 
	node[midway,below] {$A\cos\theta$}
	cycle % closed path
	% pic trick is from the angles library, requires the three points of
	% the marked angle to be named
	pic [draw,blue,angle radius=1cm,pic text=$\theta$,
	angle eccentricity=1.3] {angle=X--O--P};
	
	% right angle marker
	\draw ($(Px)-(0.3,0)$) -- ++(0,0.3) -- ++(0.3,0);
	%Point Particle drawing
	\filldraw[blue] (2.12,2.12) circle (4pt);
	
\end{tikzpicture}
\end{center}

\textbar{Therefore, we may write the position over time as:}
\begin{center}
$ x(t)=A\cos(\omega t) \implies x(t) = A\cos(\omega t + \theta_{0})
$
\end{center}	
\textbar{Differentiating this equation twice, we find:}

\begin{center}
$ \frac{dx}{dt} = -A\omega \sin(\omega t)
$
\end{center}
\begin{center}
	$\frac{d^{2}x}{dt^2} = - A\omega^2\cos(\omega t)
	$
\end{center}

\textbar{Inserting the equation for acceleration and position about the x-axis into our differential equation yields:}

\begin{center}
$m(- A\omega^2\cos(\omega t)) + k(A\cos(\omega t)) = 0 
$
\end{center}

\begin{center}
	$- A\omega^2\cos(\omega t) + \frac{k}{m}(A\cos(\omega t)) = 0 
	$
\end{center}
\newpage
\textbar{For when $\omega = \sqrt{\frac{k}{m}}$ then:}
\begin{center}
$ x = A\cos\left(t\sqrt{\frac{k}{m}}\right) \implies x = A\cos\theta
$
\end{center}























		








\subsubsection{Unknown constants $M,$ $N$ and $T$}
\begin{center}
	\centering
	\includegraphics[width=0.5\linewidth]{../../../../../../../../../Downloads/MHM_xt}
	\label{fig:mhmxt}
\end{center}
\textit{The graph is described by following equation:}
\begin{center}
	$x(t) = Acos(\omega t + \theta_0)
	$
\end{center}
\textbar{By observing the graph, we may determine that the contant $M$ is equivalent to the amplitude of the function. The amplitude is the value which dictates the maximum value of $x(t)$. Thus, $M = A$. 
	
To uncover angular velocity $\omega$:}

\begin{center}
	$A\cos(\omega t + \theta_{0}) = A\cos(\omega(t+T)+\theta_0)
	$

$  \omega t + \theta_0 + 2\pi = \omega(t+T) + \theta_0
$

$  2\pi = \omega T
$
\end{center}
\textit{Solve, for $\omega$:}
\begin{center}
	$ \boxed{\omega = \frac{2\pi}{T}}
	$
\end{center}
\textbar{To uncover $\theta_0$:}
\begin{center}
	$ \omega = \frac{2\pi}{T} \rightarrow
$
	$ \frac{\theta_0}{t} = \frac{2\pi}{T} \rightarrow
	$
	$\theta_0 = \frac{2\pi(t)}{T}
	$
	
	$\boxed{\theta_0 = \frac{2\pi(N)}{T}}
	$
\end{center}
\subsection{Derivation of Energy in SHM }


	\begin{tikzpicture}[black!75,thick]
	
				\filldraw[blue] (13,0) circle (4pt);
		% Supporting structure
			\fill [pattern = north west lines] (3,-1.5) rectangle ++(-.2,3);
			\draw[thick] (3,-1.5) -- ++(0,3);
	
			\draw
			[
			decoration={
					coil,
					segment length = 1mm,
					amplitude = 2mm,
					aspect = 0.5,
					post length = 3mm,
					pre length = 3mm},
			decorate] (3,0) -- ++(10,0);
	\end{tikzpicture}

\textbar{In this scenario, the system's energy may be described as kinetic energy along with the spring's potential energy: }
\begin{center}
$ E = KE + U_s = \frac{1}{2}mv^2 + \frac{1}{2}k\Delta x^2
$
\end{center}
\textbar{Previously, we found functions which describe the position and velocity of a simple harmonic oscillator. Thus, let us insert these derivations into our equation for the energy of SHM:}
\begin{center}
	$ E = \frac{1}{2}m\left[-A\omega\sin(\omega t + \theta_0)\right]^2 + \frac{1}{2}k\left[A\cos(\omega t + \theta_0)\right]^2
	$
	
	$ E = \frac{1}{2}mA^2\omega^2\sin^2(\omega t + \theta_0) + \frac{1}{2}kA^2\cos^2(\omega t + \theta_0)
	$
	
	
	\textit{Since the relation of angular velocity, spring constant, and mass is: $\omega = \sqrt{\frac{k}{m}} $ Then: }
	
	$ E = \frac{1}{2}kA^2\left[\sin^2(\omega t + \theta_0) + \cos^2(\omega t + \theta_0)\right]
	$
	
	
	\textit{
		Using the trig identity, $\sin^2(\theta) + \cos^2(\theta) = 1$  \\ We can assess that the expression $\sin^2(\omega t + \theta_0) + \cos^2(\omega t + \theta_0) = 1$ Thus: }
	
	$\boxed{E = \frac{1}{2}kA^2}
	$
\end{center}

\subsubsection{The energy of a particle in SHM once $KE = 2U_s$}
\textbar{Consider the motion of a particle given by $x(t) = (25cm)\cos(25t) $, where $t$ is in seconds $(s)$. What is the first time the kinetic energy is twice the potential energy?

We observe that the amplitude $(A)$ is 25cm, and the angular velocity $(\omega)$ has a value of 25. Therefore, to receive written kinetic energy, let us take the derivative of position with respect to time:  }
\begin{center}
$ v(t) = \frac{dx}{dt} = -A\sin(\omega t + \theta_0) = -25cm(25)\sin(25t)
$
\end{center}
\textbar{Now set $KE = 2U_s$ and solve for $t$: }

\begin{center}
	$ KE = 2U_s
	$
	
	$\frac{1}{2}mv^2=2\frac{1}{2}(\omega^2m)x^2
	$
	
	$mv^2 = (\omega^2m)x^2
	$
	
	$m\left[v(t)\right]^2 = 2(\omega^2m)\left[x(t)\right]^2
	$
	
	$\left[-A\omega\sin(\omega t + \theta_0)\right]^2 = 2(\omega^2)\left[A\cos(\omega t + \theta_0)\right]^2
	$
	
	$\left[-25cm(25)\sin(25t)\right]^2 = 2(25^2)\left[(25cm)\cos(25t)\right]^2
	$
	
	$\left[-625\sin(25t)\right]^2 = 1250\left[(25cm)\cos(25t)\right]^2
    $
    
    $(390 625)\sin^2(25t) = 1250(625)\cos^2(25t)
    $
    
    $\sin^2(25t) = 2\cos^2(25t)
    $
    
    
 	\textit{Using trig identity $\sin^2(\theta) + \cos^2(\theta) = 1$, we may express $sin^2(\theta) = 1 - \cos^2(\theta)$. Thus:}
 	
 	$ 1 - \cos^2(25t) = 2\cos^2(25t)
 	$
 	
 	$ 1 = 3\cos^2(25t) 
 	$
 	
 	$ \frac{1}{3} = \cos^2(25t)
 	$
 	
 	$ \arccos\left(\frac{1}{\sqrt{3}}\right) = 25t
 	$
 	
 	$\boxed{t = \arccos\left(\frac{1}{\sqrt{3}}\right)\left(\frac{1}{25}\right) = 0.038s}
 	$
\end{center} 

\subsection{An Example of a Classical Problem in SHM}
\textbar{A block attached to a spring which undergoes simple harmonic motion. In this case, the block oscillates horizontally on a frictionless table, meaning we assume no external forces are acting upon the system. Therefore the oscillations is solely determined by the properties of the spring and the initial conditions of the block's motion. The initial conditions are as follows: }

\begin{center}
$ m = 13.0\times10^{-2}kg \hspace*{1cm} k = 2.0 N/m \hspace*{1cm}  x_0 = -5.4\times10^{-2}m \implies v = 25.0\times10^{-2}m/s
$
\end{center}
\textbar{We know energy is conserved, due to the lack of external forces. Therefore, let us use the expression we derived for the energy of SHM, to uncover the amplitude:}
\begin{center}
$ E = KE + U_s = \frac{1}{2}kA^2
$
\end{center}
\textbar{Rearrange expression for amplitude:}
\begin{center}
$E = \frac{1}{2}kA^2
$

$\frac{2E}{k} = A^2
$

$A = \sqrt{\frac{2E}{k}}
$
\end{center}
\textbar{Before solving for the amplitude, we must solve for the system's energy since it was not given in the initial conditions.}
\begin{center}
$ E = KE + U_s = \frac{1}{2}mv^2 + \frac{1}{2}kx_0^2
$

$ \frac{1}{2}(13.0\times10^{-2}kg)(25.0\times10^{-2}m/s)^2 + \frac{1}{2}(2.0N/m)(-5.4\times10^{-2}m)^2 = 69.8\times10^{-4}J
$
\end{center}
\textbar{Now we can solve for amplitude:}
\begin{center}
$\boxed{A = \sqrt{\frac{2E}{k}} = \sqrt{\frac{2(69.8\times10^{-4}J)}{2.0 N/m}} = 83.5\times10^{-3}m}
$
\end{center}
\textbar{Now that we know the amplitude is $83.5\times10^{-3}m$, we can now solve for the maximum acceleration of the block.}
\begin{center}
	
	\textit{Since the relation of angular velocity, spring constant, and mass is: $\omega = \sqrt{\frac{k}{m}} $ Then, we can solve for the maximum acceleration of the block:}
	
	$\boxed{a = A\omega^2 = A\left(\sqrt{\frac{k}{m}}\right)^2 = (83.5\times10^{-3}m)\left(\sqrt{\frac{2.0N/m}{13.0\times10^{-2}kg}}\right)^2 = 1.3 m/s^2}
	$
\end{center}
\textbar{Now let us take the second derivative to find the formula for the acceleration of the block. }
\begin{center}
$x(t) = A\cos(\omega t + \theta_0)
$

$a(t) = \frac{d^2x}{dt^2} = -A\omega^2\cos(\omega t + \theta_0)
$
\end{center}
\section{PART B} 


\end{document}
